\documentclass[dvipdfmx,uplatex,a4j,12pt]{beamer}
\usepackage{karnaugh-map}
\usetheme{JuanLesPins}
\usefonttheme{structuresmallcapsserif}
\usepackage{url}
\usepackage{listings}
\usepackage{jlisting}
\usepackage{graphicx}
\usepackage{amssymb}
\usepackage{amsthm}
\usepackage{booktabs}
\usepackage{amsmath}
\lstset{%
    basicstyle={\ttfamily\small},
    commentstyle={\ttfamily\small},
    frame=tb,
    breaklines=true,
    lineskip=-0.5ex,
    tabsize=2,
    numbers=left,
    numberstyle={\ttfamily\scriptsize},
    columns=[1]{fullflexible},
}
\AtBeginSection[]{
  \begin{frame}
  \vfill
  \centering
  \begin{beamercolorbox}[sep=8pt,center,shadow=true,rounded=true]{title}
    \usebeamerfont{title}\insertsectionhead\par%
  \end{beamercolorbox}
  \vfill
  \end{frame}
}
\renewcommand{\lstlistingname}{ソースコード}
\renewcommand\proofname{\bf 証明}
\newcommand\putsource[1]{
    \begin{frame}
        \frametitle{ソースコード}

        \lstinputlisting[caption=]{#1}
    \end{frame}
}
\newcommand\generateslides[1]{
    \include{#1}
    \putsource{#1.tex}
}
\begin{document}

\title{Beamer:\LaTeX のもう一つの使い方}
\author{toku\_san}

\begin{frame}
    \titlepage
\end{frame}

\begin{frame}{内容}
    \tableofcontents
\end{frame}

\section{はじめに}

\begin{frame}
    \frametitle{自己紹介}

    \begin{enumerate}
        \item 名前等
            \begin{itemize}
                \item toku\_san
                \item 3年 I類 コンピュータサイエンスプログラム所属
                    \item Twitter: \_toku\_san
                \item Github: toku-sa-n
            \end{itemize}
        \item 好きなもの
            \begin{itemize}
                \item Rust
                \item OS開発
                \item Gentoo
                \item Vim
                \item i3wm
                \item ラーメン
            \end{itemize}
    \end{enumerate}
\end{frame}

\begin{frame}[fragile]
    \frametitle{このスライドの公開場所}

    Githubに置いておきます.

    \begin{verbatim}
https://github.com/toku-sa-n/beamer_slide
    \end{verbatim}
\end{frame}

\section{Beamerとは}

\begin{frame}{Beamerとは}
    Beamer is a \LaTeX class for creating presentations that are held using a projector, but it can also be used to create transparency slides

    --- Beamerのユーザーマニュアルより\cite{beamer_manual}
\end{frame}

\begin{frame}
    \frametitle{Beamerの利点}

    \begin{itemize}
        \item ソースコード等を楽に載せることができる
        \item \LaTeX で書ける
    \end{itemize}
\end{frame}

\begin{frame}
    \frametitle{Beamerの欠点}

    \begin{itemize}
        \item \LaTeX で書かなければならない
    \end{itemize}
\end{frame}

\section{Beamerを使う}
\begin{frame}[fragile]
    \frametitle{プリアンブル}
    \begin{lstlisting}[language={tex}]
\documentclass{beamer}
    \end{lstlisting}
\end{frame}

\begin{frame}[fragile]
    \frametitle{タイトルページ}
    \lstinputlisting[caption=]{title.tex}
\end{frame}

\generateslides{equation}
\generateslides{code}

\begin{frame}{\lstinline{lstlisting}環境を使う場合の注意点}
    \lstinline{frame}のオプションとして\lstinline{[fragile]}を追加する必要がある.\cite{why_fragile}.
\end{frame}

\generateslides{image}
\generateslides{karnaugh-map}

\begin{frame}[fragile]
    \frametitle{\lstinline{karnaugh-map.sty}のインストール(Gentoo)}

    \lstinline{karnaugh-map.sty}$\in$\lstinline{texlive-mathscience}は,\lstinline{xstring.sty}$\in$\lstinline{texlive-latexextra}に依存している.

    \begin{lstlisting}[language={sh}, caption=\lstinline{e-file}]
%sudo emerge -avt texlive-mathscience texlive-latexextra
    \end{lstlisting}
\end{frame}

\generateslides{bib}

\end{document}
